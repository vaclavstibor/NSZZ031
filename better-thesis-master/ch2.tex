\chapter{Data}
\label{chap:data}

Newspaper articles play a key role in our web application. We must consider several essential aspects to integrate these data into our web application to ensure a smooth and effective implementation. The following chapter will discuss these aspects from different perspectives, including the programmer's viewpoint and legislative considerations. \todo{In section X.X. we will give an overview...}

For our entity-level sentiment proposes, we need a body of each article as we discuss in Chapter \ref{chap:theoretical-background} \todo{refer to chapter Related Works, where we discuss maybe why others work only with titles}. 
When selecting data source of news article, it is essential to consider several main aspects.

\begin{table}[htbp]
    \caption{Considerations for Selecting News Article Data Source}
    \centering
    \begin{tabular}{p{0.3\linewidth}p{0.5\linewidth}}
        \toprule
        \textbf{Aspect} & \textbf{Description} \\
        \midrule
        Reliability & Expresses the degree to which a source can be trusted based on its history and reputation. \\
        \hline
        Availability & Expresses the degree to which a source is available to the public. \\
        \hline
        Accessibility & Refers to the ease with which the data source can be accessed. Consider factors such as API availability, data retrieval methods, and any restrictions on accessing the news articles. \\
        \hline
        Relevance & Ensuring that the selected news articles align with the interests and requirements of your web application and its users. \\
        \hline
        Timeliness & The freshness of the news data is crucial. Choose a source that provides timely updates to keep your application's content current. \\
        \hline
        Consistency & Look for a data source that maintains a consistent format and structure, facilitating easier integration into your web application. \\
        \hline
        Licensing and Copyright & Ensure compliance with legal considerations. Verify the licensing terms and copyright issues associated with using the news articles in your application. \\
        \bottomrule
    \end{tabular}
\end{table}

\begin{description}
    \item[Reliability] Expresses the degree to which a source can be trusted based on its history and reputation.
    \item[Availability] Expresses the degree to which a source is available to the public.
    \item[Accessibility] Refers to the ease with which the data source can be accessed. Consider factors such as API availability, data retrieval methods, and any restrictions on accessing the news articles.
    \item[Relevance] Ensuring that the selected news articles align with the interests and requirements of your web application and its users.
    \item[Timeliness] The freshness of the news data is crucial. Choose a source that provides timely updates to keep your application's content current.
    \item[Consistency] Look for a data source that maintains a consistent format and structure, facilitating easier integration into your web application.
    \item[Licensing and Copyright] Ensure compliance with legal considerations. Verify the licensing terms and copyright issues associated with using the news articles in your application.
\end{description}

\section{Third party data providers}
\subsection*{API}
Application programming interface (API) třetích stran jsou dostupná v různých cenových plánech. Každý plán poskytuje odlišný rozsah přístupu k datům, který typicky spočívá v rozsahu dat, jenž jsou v rámci daného plánu dostupná. Dalším nejběžnějším omezení je maximálním počtem dotazů v rámci specifikované časové periody. Drtivá většina poskytovatelů nabízí bezlpatné plány, díky kterým může vývojář otestovat různé endpointy a ověřit, zda odpovídají požadavkům jeho aplikace.

There is always some compromise at the expense of something else (, a proto bychom naší apilikaci dokázali omezit na počet dotazů tak). V naší aplikaci bychom se dokázali omezit na počet dotazů tak, abychom mohli uvažovat i bezplatného plánu anižbychom přišli o endpointy, jenž jsou pro naši aplikaci důležité. Avšak data obsahují častokrát velké mezery. Například Alpha Ventage poskytuje vyhledávání článků na základě tickeru a možností uvedení time range, ve kterém byly články vydány.

Vypadá to, že se můžeme dotazovat pouze na články v intervvalu 5 dní, avšak tento fakt není nikde v API zaznamenán.

Vždy jě něco na úkor něčeho jiného. 

\section{The Guardian}
The Guardian is a British daily newspaper that covers American and international news for an online, global audience.
\section{Dostupnost}
