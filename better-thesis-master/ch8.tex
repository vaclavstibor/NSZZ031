\chapter{Development Documentation}
\label{chap:development-documentation}
Since the algorithms and the concepts of named entity recognition and sentiment analysis applied to each service have been thoroughly described in Chapters \ref{chap:comapny-to-symbol-linking}, \ref{chap:entity-level-sentiment-analysis}, and the architecture in Chapter \ref{chap:architecture}, this chapter will only briefly summarize the essential points. For more detailed information, we will refer to the documentations produced by pdoc\footnote{\href{https://pdoc3.github.io/pdoc/}{https://pdoc3.github.io/pdoc/}}. This documentation includes extensive functional and module comments and adheres to the PEP8\footnote{\href{https://peps.python.org/pep-0008/}{https://peps.python.org/pep-0008/}} standard. Additionally, Chapter \ref{chap:architecture} thoroughly describes the \acrshort{etl} process, covering the various steps required to obtain and prepare the data for subsequent analysis. The frontend documentation has not been generated, so we will discuss it in more depth. However, it also contains strong enough comments to describe the functionality.

Our deployment details are not included in this text because the application is available on a server provided by the university. Unified services run on the server, each handled by Docker to ensure independent functionality. Only two server ports are tunnelled, as we only need one to manage \acrshort{etl} service through the Airflow webserver and one to provide access to the frontend. The \acrshort{api} services have their OpenAPI specifications in JSON generated by Swagger\footnote{\href{https://swagger.io}{https://swagger.io}}, which can be displayed within Swagger EDITOR\footnote{\href{https://editor.swagger.io}{https://editor.swagger.io}}.

\section{Backend}
\label{sec:development-backend}

\subsection{Extract Transform Load Service}
\label{subsec:development-etl}
The extract transform load service is located in the directory \textit{/app/backend/extract-transform-load}.

\begin{itemize}
    \item \textbf{Documentation}: \textit{/docs/documentation/dev/etl}.
    \item \textbf{Airflow Webserver} is availible on server\footnote{Contact me for more information.}, use Viewer account 
    \begin{itemize}
        \item \textbf{login}: \textit{viewer}
        \item \textbf{password}: \textit{viewer}
    \end{itemize}
    \item \textbf{Main Directories}
    \begin{itemize}
        \item \textbf{\textit{/airflow}} - contains the Airflow \acrshort{dag} loading script to the Airflow and the configuration files
        \item \textbf{\textit{/dag}} - contains the \acrshort{etl}'s \acrshort{dag}, detailed in attached documentation
        \item \textbf{\textit{/data}} - contains the \acrshort{etl} phases data checkpoints
        \item \textbf{\textit{/scripts}} - contains Docker scripts for the Airflow and Postgres initialisation.
    \end{itemize}
\end{itemize}

\subsection{Named Entity Recognition Service}
\label{subsec:development-ner}
The named entity recognition service is located in the directory \textit{/app/backend/named}\textit{entity-recognition}.

\begin{itemize}
    \item \textbf{Documentation}: \textit{/docs/documentation/dev/ner}.
    \begin{itemize}
        \item \textbf{OpenAPI}: \textit{openapi.json}
    \end{itemize}
    \item \textbf{Main Directories} are detailed described in the documentation.
    \item \textbf{Spacy Model}: \textit{en\_core\_web\_md}
    \item \textbf{Spacy Entity Linker Model}: version $1.0.3$
\end{itemize}

\subsection{Sentiment Analysis Service}
\label{subsec:development-sentiment}
The sentiment analysis service is located in the directory \textit{/app/backend/sentiment-analysis}.

\begin{itemize}
    \item \textbf{Documentation}: \textit{/docs/documentation/dev/sentiment}.
    \begin{itemize}
        \item \textbf{OpenAPI}: \textit{openapi.json}
    \end{itemize}
    \item \textbf{Main Directories} are detailed described in the documentation.
    \item \textbf{Analysis Model}: \textit{amphora/FinABSA-Longer}
\end{itemize}

\section{REST API Service}
\label{sec:development-rest-api}
The \acrshort{restapi} service is located in the directory \textit{/app/backend/rest-api}.

\begin{itemize}
    \item \textbf{Documentation}: \textit{/docs/documentation/dev/rest-api}.
    \begin{itemize}
        \item \textbf{OpenAPI}: \textit{openapi.json}
    \end{itemize}
    \item \textbf{Main Directories} are described in the documentation, within resources and schemata, and in OpenAPI documentation.
    \item \
\end{itemize}

\section{Frontend}
\label{sec:development-frontend}
The frontend service is located in the directory \textit{/app/frontend}. The frontend components are structured in the following way:

\subsection{Components Overview}
\label{subsec:development-components-overview}
The frontend components are structured in the following way:

\subsubsection{HomeComponent}
\label{subsubsec:development-home}
The \textit{HomeComponent} is a simple component with no data mapping. It is dedicated to displaying the home page.

\subsubsection{CompaniesComponent}
\label{subsubsec:development-companies}
The \textit{CompaniesComponent} is dedicated to searching for companies using the \textit{CompanyName} model as a list, which is directly mapped to data from the \acrshort{restapi} endpoint \textit{/api/v0/companies/names}.

\subsubsection{DashboardComponent}
\label{subsubsec:development-dashboard}
The \textit{DashboardComponent} is a parent component fetching data for its child components. While data are fetching for each child component separately, the data loading bar is displayed within the separate children. The child components are as follows:

\begin{itemize}
    \item \textbf{ArticleListComponent} - component dedicated to displaying all articles as a list. It is directly mapped to data from the \acrshort{restapi} endpoint \textit{/api/v0/company/}-\textit{\{ticker\}/articles} using the \textit{CompanyArticlesList} model.
    
    \item \textbf{CompanyInfoComponent} - component dedicated to displaying company information. It is directly mapped to data from the REST:
    \begin{itemize}
        \item \textbf{Endpoint}: \textit{/api/v0/company/\{ticker\}/info}  
        \item \textbf{Model}: \textit{CompanyInfo}
    \end{itemize}
    
    \item \textbf{PieChartComponent} - component dedicated to displaying the company's daily average sentiment value distribution in a pie chart. It is directly mapped to data from the \acrshort{restapi} :
    
    \begin{itemize}
        \item \textbf{Endpoint}: \textit{/api/v0/company/ticker/chart}  
        \item \textbf{Model}: \textit{CompanyChart}
    \end{itemize}
    
    \item \textbf{SplineChartComponent} - component displaying company sentiment value evolution in a spline chart. It is directly mapped to data from the \acrshort{restapi}:
    \begin{itemize}
        \item \textbf{Endpoint}: \textit{/api/v0/company/\{ticker\}/chart}  
        \item \textbf{Model}: \textit{CompanyChart}
    \end{itemize}

    \item \textbf{StockChartComponent} - component dedicated to displaying company price and sentiment. It is directly mapped to data from the \acrshort{restapi}:
    \begin{itemize}
        \item \textbf{Endpoint}: \textit{/api/v0/company/\{ticker\}/chart}  
        \item \textbf{Model}: \textit{CompanyChart}
    \end{itemize}
\end{itemize}

\subsubsection{GraphsComponent}
\label{subsubsec:development-graphs}
The \textit{GraphsComponent} is a component dedicated to displaying all companies and articles as one graph. The components are separated into \textit{Graph} and \textit{Graphs} due to the ability to customise the graph and display the data so that the user can easily understand the data. It is directly mapped to data from the \acrshort{restapi}:

\begin{itemize}
    \item \textbf{Endpoint}: \textit{/api/v0/companies/graphs}
    \item \textbf{Model}: \textit{CompaniesGraphs}
\end{itemize}

\subsection{Services Overview}
\label{subsec:development-services-overview}
The frontend services provide communication with the \acrshort{restapi}. Using the \textit{HttpClient} module requesting GET methods to the endpoints using dedicated models to ensure data consistency.

\subsection{Routes}
\label{subsec:development-routes}
The frontend routes manage navigation through the application. It uses the \textit{RouterModule} to navigate between components. If \textit{ticker} is in the URL, the application will automatically load the company data related to the company with the given \textit{ticker}.

\begin{itemize}
    \item \textbf{HomeComponent}: \textit{/} or \textit{/home}
    \item \textbf{CompaniesComponent}: \textit{/companies}
    \item \textbf{GraphsComponent}: \textit{/companies/graphs}
    \item \textbf{GraphComponent}: \textit{/company/:ticker/graph}
    \item \textbf{DashboardComponent}: \textit{/company/:tickerdashboard}
\end{itemize}

