
\chapwithtoc{Conclusion}
This thesis aimed to provide users with a tool that allows them to monitor and analyse the flow of information about emotional overtones as a fundamental aspect of market decisions. This objective was achieved by developing an application that extracts data from the Guardian, a British daily newspaper, analyses the sentiment of the extracted entities and provides this data as an indicator of potential future influences on a company's stock price. Acquiring reliable data made the development demanding, as obtaining data directly from providers rather than third-party sources is more challenging.

We investigated the most suitable approach for extracting company entities and their tickers. Building on the Spacy Entity Linker library, we demonstrated a flexible method for extracting information regarding a given entity. Using our approach to entity recognition and the subsequent employment of the FinABSA model, we achieved a $92\%$ success rate in sentiment evaluation when testing the reduced FinEntity dataset, primarily due to FinABSA. We then provided a short demonstration of using sentiment within the stock market.

Users have continuous access to ensure reliable data results, even during updates and processing. The application prioritises accuracy and reliability by minimising data noise. The application also allows users to verify and test historical data to see how our algorithms evaluated previous sentiment data and how the market reacted, helping them consider the values our application will evaluate. Additionally, it is adaptable to various sources, expanding the range of companies it can cover.

We designed graph network visualisation, allowing users to visualise connections between companies and news articles. The impact of news sentiment on a company's stock price is also visualised within the stock chart, supported by spline and pie charts displaying additional analyses. The application is designed to be user-friendly and intuitive, allowing users to navigate to the articles and perform their analysis to verify the values provided by the application. It is also organised with a modular architecture, enabling the manageable integration of new analysis components or models through its multi-service structure. Both of the application's leading algorithms were tested on historical data. 

In addition, we acknowledge that the sentiment analysis algorithm could be faster, but this is primarily due to the FinABSA model, which takes much time due to the length of textual data. However, this is currently sufficient as the model evaluates the data with impressive accuracy, and there is enough time between \acrshort{etl} intervals to execute it, allowing us to provide results anytime. New types of data sources, such as social media, could be integrated to gain additional sentiment insights, or other text data sources covering publicly traded companies could be included.

%This thesis aimed to provide users with a tool that allows them to monitor and analyse the flow of information about emotional overtones as a fundamental factor in market decisions. It was achieved by developing an application that extracts data from news articles, analyses the sentiment of the extracted entities, and provides this information as an indicator of potential future influences on a company's stock price. A graph network allows users to visualise connections between companies and news articles. The impact of news sentiment on a company's stock price is also visualised within the stock chart, supported by spline and pie charts displaying additional analyses about sentiment. The application is designed to be user-friendly and intuitive, allowing users to navigate to the articles and perform their analysis to verify the values provided by the application.

%We chose the entity level as a suitable approach for sentiment analysis, developing a reliable algorithm to extract entities with their ticker symbols for subsequent sentiment analysis. Both of the application's leading algorithms were tested on historical data. The application allows users to verify and test historical data to see how our algorithms evaluated previous sentiment data and how the market reacted, helping them consider the values our application will evaluate.

%The application is designed with a modular architecture, enabling the seamless integration of new analysis components or models through its multi-service structure. Users have continuous access to data, even during updates and processing. The core news data is consistent, ensuring dependable results. The application prioritises accuracy and reliability by minimising data noise. Additionally, it is adaptable to various sources, expanding the range of companies it can cover.

%In addition, we acknowledge that the sentiment analysis algorithm could be faster, but this is primarily due to the FinABSA model, which takes much time due to the volume of textual data. However, this is currently sufficient as the model evaluates the data with impressive accuracy, and there is enough time between \acrshort{etl} intervals to execute it, allowing us to provide results anytime. New data sources, such as social media, could be integrated to gain additional sentiment insights, or other text data sources covering publicly traded companies could be included. 