
\chapwithtoc{Introduction}

Nowadays with (In today's era of) information explosion and constant flow of information, it becomes time-consuming to keep track of associations and understand the content that is being disseminated through media and online news. Sentiment analysis, the ability to identify and evaluate the emotional charge of content, has become a key tool for understanding opinions, attitudes, and the general atmosphere surrounding various topics. This work focuses on the development of an application that allows users to visualize and analyze the sentiment of news in real time, using the concept of a knowledge graph network. 

Knowledge grafy se stávají stále populárnějším nástrojem pro reprezentaci a propojení informací. V naší aplikaci jsou využity k tomu, aby uživatelům umožnily nejen sledovat sentiment jednotlivých zpráv, ale také propojovat a vizualizovat vztahy mezi společnostmi prostřednictvím článků. Tímto způsobem mohou uživatelé získat holistický pohled na dění ve světě a porozumět interakcím mezi různými subjekty(?).

V práci se budeme zabývat jak technickými aspekty analýzy sentimentu, tak i návrhem a implementací aplikace, která tyto informace přenáší uživatelům co nejefektivněji. Cílem je poskytnout uživatelům nástroj, který jim umožní nejen pasivně konzumovat zprávy, ale aktivně sledovat a analyzovat tok informací s ohledem na emocionální podtón, a to v reálném čase.



Introduction should answer the following questions, ideally in this order:
\begin{enumerate}
\item What is the nature of the problem the thesis is addressing?
\item What is the common approach for solving that problem now?
\item How this thesis approaches the problem?
\item What are the results? Did something improve?
\item What can the reader expect in the individual chapters of the thesis?
\end{enumerate}

Expected length of the introduction is between 1--4 pages. Longer introductions may require sub-sectioning with appropriate headings --- use \texttt{\textbackslash{}section*} to avoid numbering (with section names like `Motivation' and `Related work'), but try to avoid lengthy discussion of anything specific. Any ``real science'' (definitions, theorems, methods, data) should go into other chapters.
\todo{You may notice that this paragraph briefly shows different ``types'' of `quotes' in TeX, and the usage difference between a hyphen (-), en-dash (--) and em-dash (---).}

It is very advisable to skim through a book about scientific English writing before starting the thesis. I can recommend `\citetitle{glasman2010science}' by \citet{glasman2010science}.
